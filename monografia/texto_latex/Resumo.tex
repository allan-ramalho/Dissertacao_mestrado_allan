% Keywords command
\providecommand{\keywords}[1]
{
	\small	
	\textbf{\textit{Palavras chave: }} #1
}
\begin{center}
{\Huge Resumo}
\end{center}	
Neste trabalho, foi realizado um estudo comparativo entre dois conjuntos de dados de distúrbio de gravidade sobre a região da Bacia do Parnaíba: o primeiro refere-se aos distúrbios terrestres adquiridos pelo Observatório Nacional; o segundo é proveniente do modelo de gravidade global \textit{EIGEN-6C4}, onde um compilado de dados de gravidade terrestres, aéreos, marinhos e de satélite são utilizados no cálculo dos coeficientes da expansão em harmônicos esféricos. Os levantamentos gravimétricos são extremamente importantes para a geofísica aplicada, por fornecerem meios de avaliar e inferir informações relevantes sobre a distribuição de densidade interna da Terra. Esses levantamentos diferem na forma de aquisição, seja terrestre ou transportada, e na cobertura de dados. A depender do acesso à regiões remotas, a gravimetria terrestre pode ser proibitiva, o que acarreta em uma limitação na cobertura espacial dos dados. Como alternativa para essa limitação surge a gravimetria transportada. No entanto, os dados obtidos nesse tipo de aquisição podem apresentar ruídos indesejáveis por consequência das oscilações e vibrações dos veículos empregados. A utilização de um modelo global supre tais limitações, uma vez que esses modelos permitem calcular valores de gravidade em qualquer ponto do globo. Dessa forma, é possível descrever o campo de gravidade em qualquer ponto da Terra. Apesar disso, sua aplicação requer uma análise mais cuidadosa, devido às limitações associadas ao trucamento da série harmônica e às grandes distâncias entre os satélites e a subsuperfície investigada. Neste trabalho foi apresentada uma análise qualitativa baseada na comparação entre gravimetria terrestre e  de satélite. Para tal, foi calculado o vetor de resíduos entre os distúrbios de gravidade terrestres e os calculados via modelo, chamados aqui de observados e preditos, respectivamente. Os histogramas dos resíduos apresentaram média positiva e desvio padrão relativamente elevados, o que indica uma subestimativa dos distúrbios de gravidade. Isso pode estar relacionado à baixa amostragem dos dados utilizados neste trabalho. Outro potencial motivo consiste da não inclusão de dados terrestres e aéreos ao modelo na região estudada, o que acarreta em um aumento das feições irrealistas nos distúrbios preditos. Um terceiro fator bastante relevante é o truncamento da série harmônica, que pode ser decisivo para justificar os resíduos apresentados neste estudo.


\noindent{\keywords{gravimetria; distúrbio de gravidade; modelo de gravidade global.}}




