\chapter{Introdução}

A gravimetria é um método geofísico que tem por objetivo determinar a aceleração da gravidade da Terra sobre ou próximo à sua superfície \cite{torge1989}. Dentro desse escopo, os levantamentos gravimétricos contribuem significativamente para estudos geológicos e geofísicos, pois geram meios de avaliar e inferir sobre a distribuição de densidade interna da Terra. O Observatório Nacional (ON) é a instituição brasileira pioneira em gravimetria e, em 1978, passou a ser responsável pela implantação e manutenção da Rede Gravimétrica Fundamental Brasileira (RGFB). Dessa forma, desde o fim da década de 70, o ON opera e mantém em todo o país a RGFB, disponibilizando um conjunto de estações gravimétricas terrestres de alta precisão que servem de apoio para os levantamentos gravimétricos conduzidos no Brasil \cite{escobar1987rede,subiza2001,castro2018}. A partir da segunda metade do século XX, começou a ser implementada a gravimetria transportada, aérea e náutica, que possibilitavam adquirir os dados em lugares remotos e de acesso mais restrito à gravimetria terrestre \cite{halpenny1995airborne,bell1999airborne}. De acordo com \citeonline{carbonari2007}, as aquisições aéreas e marinhas tem a vantagem de adquirir um maior conjunto de dados com uma velocidade consideravelmente maior do que no caso terrestre, podendo cobrir uma vasta área de investigação. No entanto, as oscilações verticais e horizontais causadas pela movimentação dos veículos podem gerar ruídos indesejáveis nas medidas. 

Em 1976, foi lançado o primeiro satélite de estudo geodinâmico com o objetivo de determinar o formato da Terra, iniciando a era dos satélites gravitacionais. Esse satélite, denominado LAGEOS (\textit{Laser Geodynamics Satellite}), consiste em uma esfera metálica de $60$ centímetros de diâmetro com $426$ refletores a laser e nenhum instrumento a bordo. Foi posicionado em uma órbita de $5900$ quilômetros de altitude e fornece medições da reflexão dos feixes de luz emitidos da Terra ao longo do tempo. Posteriormente, em 2000, foi lançada a missão CHAMP (\textit{Challenging Minisatellite Payload}), a primeira das modernas missões de satélites que possibilitou um detalhamento sem precedentes do campo gravitacional e magnético terrestre \cite{lobianco2005}, através de instrumentos embarcados como acelerômetro, magnetômetro, sensor inercial, retrorrefletor de raios laser e receptor  $GNSS$ (\textit{Global Navigation Satellite System}). De acordo com o mesmo autor, o satélite esteve em órbita a uma altitude inicial de $454$ quilômetros, e, visando realizar estudos mais densos sobre o campo de gravidade, teve sua altitude alterada para $300$ quilômetros. Foi a primeira vez que um satélite com receptores GNSS orbitou a baixas altitudes, permitindo que o seu posicionamento ocorresse com a precisão de poucos centímetros \cite{schwintzer2002}. Em 2002, foi lançada a missão GRACE (\textit{Gravity Recovery and Climate Experiment}), composta por dois satélites gêmeos do tipo CHAMP, desenvolvida para o estudo das variações temporais do campo gravitacional da Terra \cite{torge2001geodesy,schwintzer2002}. Sua órbita inicial era de aproximadamente $500$ quilômetros. Mais tarde, em 2009, foi lançada a missão GOCE (\textit{Gravity Field and Steady-State Ocean Circulation Explorer}), a terceira das modernas missões satelitais, que teve como objetivo determinar o geoide com precisão de $\approx 2$ centímetros \cite{torge2001geodesy} e distúrbio de gravidade $\approx 2$ mGal \cite{junior2019evoluccao}. O projeto previa que o satélite estivesse em órbita com a menor altitude possível, a uma altura inicial aproximada de $260$ quilômetros e, posteriormente, de $235$ quilômetros. O principal instrumento da missão GOCE era o gradiômetro, que consistia num conjunto de acelerômetros de três eixos, que chegava a ser $100$ vezes mais sensível que os demais utilizados até então. 

As aquisições gravimétricas terrestres, apesar da resolução bastante acurada, apresentam dificuldades para serem realizadas em regiões de difícil acesso e em alguns casos podem ser inviáveis. Já as aquisições marinhas e aerotransportadas contornam esse tipo de problema. No entanto, possuem custo relativamente alto e apresentam o problema de serem utilizadas sobre plataformas móveis que influenciam diretamente nos dados \cite{carbonari2007, castro2018}. As aquisições de satélites apresentam cobertura global, amostragem relativamente alta, são regularmente espaçadas e sua utilização permite calcular a gravidade em qualquer ponto do globo. Os dados de gravidade adquiridos, através das diferentes formas possíveis, contribuem para a interpretação geofísica. Segundo \citeonline{hackney2003}, a diferença da gravidade real da Terra e a gravidade teórica resulta em anomalias e distúrbios de gravidade, sendo os conceitos de distúrbio mais difundidos sob a perspectiva geodésica e os conceitos de anomalia sob a perspectiva geofísica. Por exemplo, \citeonline{carbonari2007} mostra a utilização de anomalias de gravidade para conhecimento geológico da bacia Sergipe-Alagoas e sua aplicabilidade, na exploração de hidrocarbonetos, a fim de delimitar as principais feições estruturais, ajudando a compreender a evolução da bacia. Já \citeonline{disturbios_investig} utilizou distúrbios de gravidade para auxiliar no processo de investigação geofísica da interface entre crostas oceânica e continental na Elevação do Ceará. Para isso, é utilizada a teoria da inversão e a modelagem usando prismas justapostos. \citeonline{uieda2017} utilizou os distúrbios de gravidade na estimativa da descontinuidade de Mohorovičić (Moho) na região da América do Sul. \citeonline{marcela} utilizou distúrbios de gravidade para estimar simultaneamente as geometrias do embasamento e do Moho, bem como a profundidade do Moho de referência ao longo de um perfil que cruza uma margem passiva estriada da Bacia de Pelotas.

% Falar como se obtem os modelos de gravidade(nada extenso, colocar uma referencia)
Uma das mais relevantes atribuições de um geofísico é ser capaz de reproduzir um conjunto de dados. Dessa forma, muitas respostas podem ser obtidas de forma rápida e eficiente. Logo, a modelagem gravimétrica apresenta-se como uma poderosa ferramenta de trabalho. Em escala global, é comum a utilização dos harmônicos esféricos e funções \textit{spline} \cite{sandwell1987biharmonic}. Modelos de gravidade global (MGGs) descrevem de forma aproximada o campo de gravidade da Terra em um espaço tridimensional, através de expansões em harmônicos esféricos \cite{barthelmes2009}. De acordo com \citeonline{barthelmes2014global}, a aplicação de dados de satélites artificiais nos MGGs permitiu um significativo aprimoramento destes modelos. Os dados de satélites adquiridos principalmente através das missões LAGEOS, CHAMP, GRACE e GOCE contribuíram para o desenvolvimento de vários MGGs, marcando o início do século XXI como a década dos potenciais de gravidade \cite{barthelmes2014global,junior2019evoluccao}. Cada uma destas missões teve sua contribuição específica em termos de coeficientes de um MGG. Os dados do LAGEOS contribuíram para os baixos graus dos modelos, em torno de 5 \cite{torge2001geodesy}. De acordo com o mesmo autor, os dados da missão GRACE seriam utilizados para modelar a Terra no que diz respeito aos médios graus, até 150. Já as informações do GOCE modelam o planeta para os graus mais altos, em torno de 300 \cite{torge2001geodesy,forste2016}. Segundo \citeonline{reigber2003champ}, os dados obtidos na missão CHAMP deram origem a um MGG quatro vezes mais preciso que os anteriores, o \textit{EIGEN-2}, que foi desenvolvido em termos de harmônicos esféricos até grau e ordem $120$. Segundo \citeonline{junior2019evoluccao}, o primeiro MGG  com dados obtidos na missão GRACE apresentou resultados de 10 a 50 vezes mais precisos do que qualquer outro  MGG que atuasse em faixas de médios e/ou longos de comprimentos de onda. Os MGGs mais antigos expandem a série em harmônicos a grau e ordem  relativamente baixos, reflexo do desenvolvimento tecnológico e computacional de suas respectivas épocas. Já os modelos mais recentes, como o \textit{EGM2008} e o \textit{EIGEN-6C4} , apresentam valores relativamente altos de grau e ordem. Esse nível de detalhamento é atingindo por serem modelos combinados, isto é, além dos recursos computacionais e tecnológicos mais modernos, eles utilizam dados de satélites combinados com dados terrestres, aerotransportados e marinhos \cite{eigen,barbosagravimetria}.

Tendo em vista a temática abordada, o objetivo deste trabalho consiste na análise comparativa entre distúrbios de gravidade terrestre observados e preditos para a região da Bacia do Parnaíba. O conjunto de dados utilizados para elaboração dessa pesquisa são dados de altitude e de gravidade terrestres, que foram adquiridos e disponibilizados pelo ON, e dados modelados através do MGG \textit{EIGEN-6C4}. Para fins comparativos, utilizamos os mesmos pontos de observação para estimar altitudes e gravidades absolutas através do modelo disponível no ICGEM (International Centre for Global Earth Models). As altitudes geométricas foram utilizadas para calcular a gravidade normal, através da fórmula analítica da gravidade de \citeonline{li2001}. Posteriormente, calculam-se os distúrbios de gravidade pela diferença entre as gravidades absoluta e normal sob cada coordenada do levantamento terrestre. Visando complementar o estudo, os distúrbios preditos e observados foram comparados através de uma análise estatística, por meio de histogramas dos resíduos e gráficos de dispersão, tanto das altitudes geométricas quanto dos distúrbios de gravidade. 
