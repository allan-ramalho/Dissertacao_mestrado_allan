% Keywords command
\providecommand{\keywordss}[1]
{
	\small	
	\textbf{\textit{Keywords: }} #1
}
\begin{center}
	{\Huge Abstract}
\end{center}
In this work, a comparative study was carried out between two sets of gravity disturbance data on the Parnaíba Basin region: the first refers to terrestrial disturbances acquired by the National Observatory; the second comes from the global gravity model \textit{EIGEN-6C4}, where a compilation of terrestrial, aerial, marine and satellite gravity data is used to calculate the expansion coefficients in spherical harmonics. Gravimetric surveys are extremely important for applied geophysics, as they provide a means of evaluating and inferring relevant information about the Earth's internal density distribution. These surveys differ in the form of acquisition, whether overland or transported, and in data coverage. Depending on access to remote regions, terrestrial gravimetry can be prohibitive, which results in a limitation in the spatial coverage of the data. As an alternative to this limitation, transported gravimetry appears. However, the data obtained in this type of acquisition can present undesirable noises due to the oscillations and vibrations of the vehicles used. The use of a global model overcomes these limitations, since these models allow the calculation of gravity values anywhere in the globe. In this way, it is possible to describe the gravity field at any point on Earth. Despite this, its application requires a more careful analysis, due to the limitations associated with the harmonic series truncation and the great distances between the satellites and the investigated subsurface. In this work, a qualitative analysis based on the comparison between terrestrial and satellite gravimetry was presented. For this purpose, the vector of residuals was calculated between terrestrial gravity disturbances and those calculated via the model, called here observed and predicted, respectively. The residual histograms showed a relatively high positive mean and standard deviation, which indicates an underestimation of the gravity disturbances. This may be related to the low sampling of data used in this work. Another potential reason is that land and air data are not included in the model in the studied region, which leads to an increase in unrealistic features in the predicted disturbances. A third very relevant factor is the truncation of the harmonic series, which can be decisive to justify the residues presented in this study.

\noindent{\keywordss{gravimetry; gravity disturbance; global gravity model.}}	





